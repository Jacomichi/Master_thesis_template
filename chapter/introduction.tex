\documentclass[../thesis]{subfiles}
% このファイル内だけのコマンド
\begin{document}
\chapter{序論}

\section{研究背景}
平衡系ではxxxがよく知られている~\cite{zwanzig2001nonequilibrium}。
一方で、非平衡系ではyyyとなることが知られている~\cite{Widder1989,Stolovitzky1998}。

physicsパッケージを使って数式をかける。
\begin{equation}
\pdv{\rho}{t} = -\nabla \cdot \bm{J}
\end{equation}

\section{本研究の目的と方法}

\section{本論文の構成}
本論文は以下の構成からなる。

第1章では本論文の位置づけおよび、本論文の構成について述べる.
第2章では本研究に関連する先行研究について説明を行う.
第3章では数値計算手法および、その設定について説明を行う.
第4,5章では本研究で得た主要な結果について述べる。
第6章では本研究の結論と今後の展望に関して述べる.
付録Aでは本研究の数値計算に関する予備研究について述べた.
付録Bでは解析手法について詳細な説明を行う。

\end{document}